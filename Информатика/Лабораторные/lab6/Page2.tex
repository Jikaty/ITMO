
\begingroup
\fontsize{14}{18}\selectfont

\begin{center}
    \Large
    \textbf{КИНЕМАТИКА И СВЯЗИ}
\end{center}
\begin{flushright}
    \textbf{C. А. Беляев}
\end{flushright}

\begin{center}
    \includegraphics[]{Photo}
\end{center}
\vspace{1em}
\begin{multicols}{2}
    \setlength{\columnsep}

    \begin{minipage}{8cm}
        \setlength{\parindent}{0.5cm}
        Кинематика часто рассматривает движение абсолютно твердых тел, то есть тел, расстояния между любыми двумя точками которых остаются постоянными. При этом существуют методы, значительно упрощающие решение кинематических задач. С одним из них мы сейчас познакомимся.
        \vspace{1em}
        \setlength{\parindent}{0.5cm}
        
        Пусть тела при движении соприкасаются, и скольжение между ними отсутствует. Тогда скорости обоих тел в точке соприкосновения полностью совпадают (рис. 1). Если же между телами есть проскальзывание, то совпадают лишь проекции скоростей на перпендикуляр к касательной в точке соприкосновения. При этом достаточно, чтобы касательная существовала хотя бы для одной из скользящих поверхностей (рис. 2).
        \begin{flushright}
            Рассмотрим несколько примеров.\\
            1. Стержень $OA$ вращается по ча-
        \end{flushright}
    \end{minipage}

    \begin{minipage}{8cm}
        совой стрелке с угловой скоростью $\omega$, приводя в движение кирпич $ABCD$ с боковой стороной $a$ (рис. 3). Найти зависимость скорости кирпича $v$ от угла $\alpha$.
        \vspace{1em}
        \setlength{\parindent}{0.5cm}
        
        Р е ш е н и е. Стержень и кирпич соприкасаются в точке $A$. Следовательно, скорости кирпича и стержня в этой точке в направлении $MN$ ($MN \perp OA$) совпадают. Таким образом,
        $$v \cos (90^\circ - \alpha) = \omega \cdot OA,$$или$$v = \frac{\omega}{\sin \alpha} \cdot \frac{a}{\operatorname{tg} \alpha} = \frac{\omega a \cos \alpha}{\sin^2 \alpha}.$$
        \vspace{1em}
        \setlength{\parindent}{0.5cm}

        2. Источник света $S$ находится на расстоянии $l$ от экрана $MN$ (рис. 4). В начальный момент времени плоский предмет высоты $h$ начинает равномерно двигаться со скоростью $v$ от источника к экрану. Найти зависимость скорости движения края тени по экрану от времени.
    \end{minipage}   
    
    

\end{multicols}






\endgroup


